\RequirePackage{plautopatch}
\documentclass[dvipdfmx]{beamer}

\usetheme{Copenhagen}
\usepackage{graphicx}
\setbeamertemplate{navigation symbols}{}
%\renewcommand{\kanjfamilydefault}{\gtdefaut}
%\usefonttheme{professionalfonts}
%\usepackage{deluxe}{otf}
%\usepackage[noalphabet]{pxchfon}
%\setboldgothicfont{HaranoAjiGothic-Medium.otf}
%\usepackage[format=plain, labelformat=simple, labelse=period]
\usepackage{amsmath}
\usepackage{siunitx}
%\usepackage{calculation}
\usepackage{pgffor}
\usepackage{ifthen}
\usepackage{fp}
\usepackage[export]{adjustbox}
%\usepackage{mathtool}

%\documentclass{article}

\newcommand{\zeropad}[1]{%
  \ifnum#1<10 0#1\else #1\fi%
}

\newcommand{\generateframe}[2]{

  \begin{frame}
    \frametitle{結果}
    \def\filefig{../figure/\zeropad{#1}_\zeropad{#2}_fig.png}
    \begin{columns}[T]
        \includegraphics[width=\linewidth]{\filefig}
    \end{columns}
  \end{frame}
}


\makeatletter

\title{
  Stimberg et al., 2019 再現実装\
}

\begin{document}

\begin{frame}
  \titlepage
\end{frame}


%1st page
\begin{frame}{シミュレーション条件}
  \begin{itemize}
    \item ニューロンモデルは、Leaky integrate-and fire model
    \item シナプスは、Tsodyks-Markram model
    \item 古典的なバランスネットワーク(Brunel., 200)と類似したダイナミクスを持つ。
    \item 詳細は、Stimberg et al., 2019のFigure 1.を参照。
  \end{itemize}
\end{frame}

\foreach \G_EXC_index in {0,...,10} {
  \foreach \G_INH_index in {0,...,10} {
    \generateframe{\G_EXC_index}{\G_INH_index}
  }
}
\end{document}

